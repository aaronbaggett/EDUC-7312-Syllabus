% Options for packages loaded elsewhere
\PassOptionsToPackage{unicode}{hyperref}
\PassOptionsToPackage{hyphens}{url}
%
\documentclass[
]{article}
\usepackage{lmodern}
\usepackage{amssymb,amsmath}
\usepackage{ifxetex,ifluatex}
\ifnum 0\ifxetex 1\fi\ifluatex 1\fi=0 % if pdftex
  \usepackage[T1]{fontenc}
  \usepackage[utf8]{inputenc}
  \usepackage{textcomp} % provide euro and other symbols
\else % if luatex or xetex
  \usepackage{unicode-math}
  \defaultfontfeatures{Scale=MatchLowercase}
  \defaultfontfeatures[\rmfamily]{Ligatures=TeX,Scale=1}
  \setmainfont[]{Times New Roman}
  \setmonofont[]{Lucida Console}
\fi
% Use upquote if available, for straight quotes in verbatim environments
\IfFileExists{upquote.sty}{\usepackage{upquote}}{}
\IfFileExists{microtype.sty}{% use microtype if available
  \usepackage[]{microtype}
  \UseMicrotypeSet[protrusion]{basicmath} % disable protrusion for tt fonts
}{}
\makeatletter
\@ifundefined{KOMAClassName}{% if non-KOMA class
  \IfFileExists{parskip.sty}{%
    \usepackage{parskip}
  }{% else
    \setlength{\parindent}{0pt}
    \setlength{\parskip}{6pt plus 2pt minus 1pt}}
}{% if KOMA class
  \KOMAoptions{parskip=half}}
\makeatother
\usepackage{xcolor}
\IfFileExists{xurl.sty}{\usepackage{xurl}}{} % add URL line breaks if available
\IfFileExists{bookmark.sty}{\usepackage{bookmark}}{\usepackage{hyperref}}
\hypersetup{
  hidelinks,
  pdfcreator={LaTeX via pandoc}}
\urlstyle{same} % disable monospaced font for URLs
\usepackage[margin=1in]{geometry}
\usepackage{graphicx}
\makeatletter
\def\maxwidth{\ifdim\Gin@nat@width>\linewidth\linewidth\else\Gin@nat@width\fi}
\def\maxheight{\ifdim\Gin@nat@height>\textheight\textheight\else\Gin@nat@height\fi}
\makeatother
% Scale images if necessary, so that they will not overflow the page
% margins by default, and it is still possible to overwrite the defaults
% using explicit options in \includegraphics[width, height, ...]{}
\setkeys{Gin}{width=\maxwidth,height=\maxheight,keepaspectratio}
% Set default figure placement to htbp
\makeatletter
\def\fps@figure{htbp}
\makeatother
\setlength{\emergencystretch}{3em} % prevent overfull lines
\providecommand{\tightlist}{%
  \setlength{\itemsep}{0pt}\setlength{\parskip}{0pt}}
\setcounter{secnumdepth}{5}
\usepackage{fancyhdr, lastpage}
\pagestyle{fancy}
% \usepackage{graphicx}
% \graphicspath{{figs/}}
\usepackage{enumitem}
\usepackage{setspace}
\usepackage{verbatim}
\usepackage{hanging}
\usepackage{multirow}
\usepackage{float}
\usepackage{lscape}
\usepackage{tabulary}
\usepackage{tabularx}
\usepackage{tcolorbox}
\usepackage{array}
\newcolumntype{H}{>{\setbox0=\hbox\bgroup}c<{\egroup}@{}}

\usepackage{hyperref}
\hypersetup{  
   colorlinks,   
   citecolor=blue,
   linkcolor=blue,
   urlcolor=blue,
}

\usepackage{color, colortbl}
\definecolor{Gray}{gray}{0.9}

\renewcommand{\headrulewidth}{0pt}

\chead{}
\rhead[p. \thepage\ of\ \pageref*{LastPage}]{p. \thepage\ of\ \pageref*{LastPage}}
\lhead{}

\fancyfoot{}
\fancyfoot[L]{\textcolor{gray}{EDUC 7312-01}}
\fancyfoot[C]{\textcolor{gray}{Fall 2020}}
\fancyfoot[R]{\textcolor{gray}{Aaron R. Baggett, Ph.D.}}


\author{}
\date{\vspace{-2.5em}}

\begin{document}

\makeatletter
\setlength{\@fptop}{0pt}
\makeatother

\begin{figure}[t!]
  \centering
  \includegraphics[keepaspectratio, width=0.15\textwidth]{./Figs/Seal-Black}
\end{figure}


\begin{center}
{\LARGE{\bf{EDUC 7312-01 Research III Advanced Design and Methods}}}\\
% \vspace{.5mm}
% {\Large{\bf{Dissertation I}}}\\
{\small{University of Mary Hardin-Baylor}}\\
\vspace{.15in}
{\large{Fall 2020}}
\end{center}

\thispagestyle{empty}

\vspace{3mm}

\vspace{.25in}

\section{Concact Information}

\textbf{Instructor:} Aaron R. Baggett, Ph.D.\\
\textbf{Email:}
\texttt{\href{mailto:abaggett@umhb.edu}{abaggett@umhb.edu}}\\
\textbf{Office Phone:} (254) 295-4553\\
\textbf{Office Location:} Wells 140\\
\textbf{Office Hours:} MWF: 9:00 AM--11:00 AM; MW: 1:00 -- 3:00 PM; TR:
1:00 PM -- 3:00 PM, and by
\texttt{\href{mailto:abaggett@umhb.edu}{appointment}}

\begin{tcolorbox}
[width=\linewidth, sharp corners=all, colback=white!95!red]
NOTE: All student meetings for the Fall 2020 semester will occur via Dr. Baggett's personal \texttt{\href{https://umhb.zoom.us/j/5393191651}{Zoom ID}}
\end{tcolorbox}

\subsection{Description}

The purpose of Research III is to explore both quantitative and
qualitative methods with an emphasis in data analysis. Topics include
inferential statistics, triangulation of data, and rigor. Students will
develop a research proposal. SPSS software will be used for quantitative
analysis.

\subsection{Meeting}

\textbf{Dates:}

Students enrolled in EDUC 7312: Research III Advanced Design and Methods
will meet together a total of five times throughout the Fall 2020
semester between 8:30 PM--12:00 PM in the Parker Academic Center (PAC)
room 222

\begin{enumerate}
\def\labelenumi{\arabic{enumi}.}
\tightlist
\item
  Saturday, September 12, 2020, PAC 222
\item
  Saturday, October 03, 2020, PAC 222
\item
  Saturday, October 24, 2020, PAC 222
\item
  Saturday, November 14, 2020, PAC 222
\item
  Saturday, December 05, 2020, PAC 222
\end{enumerate}

\textbf{Course Website:}
\texttt{\href{https://mycourses.umhb.edu/courses/23599}{myCourses}}

\newpage
\subsection{Advanced Academic Activity}

Doctoral courses contain appropriate advanced academic activity
reflected in the areas of content, process, and product. The advanced
activity is facilitated through the dimension of critical thinking,
synthesis and integration of materials, depth of engagement of
materials, and contribution to scholarship. The purpose of advanced
academic activity is to demonstrate a higher level of sophistication and
to emphasize separation from masters level courses.

\subsection{Course Objectives}

Upon completion of this course, you should be able to:

\begin{enumerate}
\def\labelenumi{\arabic{enumi}.}
\tightlist
\item
  Understand basic to intermediate principles of statistics needed to
  both consume and conduct educational and social science research.
\item
  Communicate and present statistical ideas clearly in oral and written
  forms using appropriate technical terms and deliver data analysis
  results sufficient for both a technical and non-technical audience.
\end{enumerate}

\subsection{Student Learning Objectives}

Upon completion of all course modules, you should be able to:

\textbf{1. Module 1: Introduction to Data}

\begin{enumerate}
\def\labelenumi{\arabic{enumi}.}
\tightlist
\item
  Identify variables as numerical and categorical and further classify
  based on the nature of the
\item
  Define associated variables as variables that show some relationship
  with one another. Further categorize this relationship as positive or
  negative association, when possible.
\item
  When describing the distribution of a numerical variable, mention its
  shape, center, and spread, as well as any unusual observations.
\item
  Identify the shape of a distribution as symmetric, right skewed, or
  left skewed, and unimodal, bimodal, multimodal, or uniform.
\end{enumerate}

\textbf{2. Module 2: Foundations for Statistical Inference}

\begin{enumerate}
\def\labelenumi{\arabic{enumi}.}
\tightlist
\item
  Identify the direction and strength of both linear and non-linear
  correlations.
\item
  Compute and interpret linear and non-linear correlation coefficients
  using SPSS.
\item
  Summarize visually and verbally the results of linear and non-linear
  correlation coefficients.
\end{enumerate}

\textbf{3. Module 3: Statistical Inference for Numerical and Categorical
Data}

\begin{enumerate}
\def\labelenumi{\arabic{enumi}.}
\tightlist
\item
  Identify occasions when a predictive model is preferred, over others,
  in educational and social science research.
\item
  Compute and interpret the slope, intercept, and response variable(s)
  in both simple and multiple linear regression models using SPSS.
\item
  Summarize visually and verbally the results of both simple and
  multiple linear regression models.
\end{enumerate}

\textbf{4. Module 4: Simple and Multiple Regression}

\begin{enumerate}
\def\labelenumi{\arabic{enumi}.}
\tightlist
\item
  Identify occasions when either between- and within-groups designs are
  preferred, over others, in educational and social science research.
\item
  Identify statistical assumptions for both one-way and factorial
  between- and within-groups designs.
\item
  Compute and interpret results of main effects, interaction effects,
  and post-hoc tests in both one-way and factorial between- and
  within-groups designs using SPSS.
\item
  Summarize visually and verbally the results of both one-way and
  factorial between- and within-groups designs.
\end{enumerate}

\begin{enumerate}
\item {\bf{Gain factual knowledge such as important terminology, classifications, methods, and trends by:}}

\begin{enumerate}[label=\Alph*.]
\itemsep2pt\parskip0pt\parsep0pt
\item Participating in lectures and in-class presentations;
\item Distinguishing between the four basic scales of measurement;
\item Distinguishing between null and alternative hypotheses;
\item Distinguishing between Type I and Type II Errors;
\item Understanding characteristics of population and sample distributions;
\item Understanding essential characteristics of descriptive statistics and procedures used for summarizing data;
\item Understanding probability and the foundations of inferential statistics;
\item Making inferences about the variability between- and within-samples according to the general linear model;
\item Distinguishing between statistical and practical significance.
\end{enumerate}

\item {\bf{Learn to apply course material to improve decision-making, problem solving, and critical thinking skills related to experimental design and statistical analysis by:}}

\begin{enumerate}[label=\Alph*.]
\itemsep2pt\parskip0pt\parsep0pt
\item[A.] Completing formal assessments involving knowledge-level, conceptual, and applied material;
\item[B.] Completing a series of individual data evaluation and analysis projects involving performing, analyzing, and interpreting statistical output;
\item[C.] Communicating research findings according to the rules of APA Style (in both written and oral form);
\item[D.] Considering the ethical issues associated with research involving human and nonhuman participants.
\end{enumerate}
\end{enumerate}

\subsection{Credit Hour(s)}

For online, hybrid, and other nontraditional modes of delivery, credit
hours are assigned based on learning outcomes that are equivalent to
those in a traditional course setting; forty-five (45) hours of work by
a typical student for each hour of credit.

\subsection{Readings}

Students are required to obtain a copy of the following required
textbook.

\begin{hangparas}{.4in}{1}
Diez, D. M.,  \c{C}etinkaya-Rundel, M., \& Barr, C. D. (2019).  {\em{OpenIntro statistics}} (4\textsuperscript{th} ed.).  OpenIntro.
\end{hangparas}

All other assigned readings will be provided on the
\texttt{\href{https://mycourses.umhb.edu/courses/23599}{course website}}
under the
\texttt{\href{https://mycourses.umhb.edu/courses/23599/files/folder/Readings}{Readings}}
tab.

\subsection{Required Software}

Any data and statistical analyses in EDAD 7312 will be conducted using
IBM\textsuperscript{\textregistered}~SPSS\textsuperscript{\textregistered}~Statistics.
During class meetings, students may access local
SPSS\textsuperscript{\textregistered}~installations on campus computers.
However, you may consider purchasing at least a 6-month license to
download and install SPSS\textsuperscript{\textregistered}~on your own
computer(s). Below are a few online educational software retailers who
offer heavily discounted versions of
IBM\textsuperscript{\textregistered}~SPSS\textsuperscript{\textregistered}~Statistics
for students.

\emph{Note}: As of \today, the latest version number is 27 However,
versions 23, 24, 25, and 26 should also be compatible for all data
analyses conducted throughout this course. Keep in mind previous
versions sell for the same price as the current version. Prior to
purchasing SPSS\textsuperscript{\textregistered}~be sure you are
selecting the versions compatible with your particular computer's
operating system (i.e., Windows, macOS, etc.).
IBM\textsuperscript{\textregistered}~SPSS\textsuperscript{\textregistered}~Statistics
is not compatible with iOS, iPadOS, or Chrome OS.

\textbf{How to purchase SSPS:}

\begin{tcolorbox}
[width=\linewidth, sharp corners=all, colback=white!95!red]
NOTE: IBM\textsuperscript{\textregistered}\ SPSS\textsuperscript{\textregistered}\ Statistics Standard GradPack will be sufficient for EDUC 7310. No need to purchase the Premium GradPack.
\end{tcolorbox}

\begin{enumerate}
\def\labelenumi{\arabic{enumi}.}
\tightlist
\item
  \texttt{\href{https://onthehub.com/spss/}{OnTheHub}}

  \begin{enumerate}
  \def\labelenumii{\arabic{enumii}.}
  \tightlist
  \item
    Select \textbf{Students} buy now.
  \item
    Scroll down until you find your computer's operating system (i.e.,
    Windows, macOS, etc.) and your desired rental duration (i.e., 06- or
    12-month).
  \item
    Add to Cart.
  \end{enumerate}
\item
  \texttt{\href{https://www.journeyed.com/products/IBM+SPSS/IBM+SPSS+Statistics}{journeyEd}}

  \begin{enumerate}
  \def\labelenumii{\arabic{enumii}.}
  \tightlist
  \item
    Scroll to approximately one-third of the way down the page and look
    for \textbf{IBM SPSS Statistics Standard Grad Pack 27.0 Academic}.
  \item
    Select the link applicable to your operating system.
  \item
    Note: only 12-month licenses are available from journeyEd.
  \end{enumerate}
\end{enumerate}

\subsection{Academic Integrity}

UMHB's policy on Classroom Expectations and Ethics will be strictly
upheld in this course. If you have not read it and all subsequent
sections, it is your responsibility to do so. You may find it online
here:
\href{http://catalog.umhb.edu/2019-2020/Graduate-Catalog/Classroom-Expectations-and-Ethics}{\texttt{Classroom Expectations and Ethics}}.
The omnibus policy outlines University requirements concerning Christian
citizenship, students' responsibilities, class attendance, academic
decorum, and academic integrity.

\subsection{Disabled Student Services and Accomodations}

It is the student's responsibility to request disability accommodations.
Students requesting an accommodation for a disability, must contact the
UMHB
\href{http://cths.umhb.edu/disability}{\texttt{Counseling, Testing \& Health Services}}
as early as possible in the term.
\href{http://catalog.umhb.edu/en/2019-2020/Graduate-Catalog}{\texttt{The Course Catalog}},
\href{http://students.umhb.edu/student-handbook}{\texttt{Student Handbook}}
and \href{https://go.umhb.edu/}{\texttt{UMHB website}} provide more
details regarding the process by which accommodation requests will be
reviewed.

For more information, please contact:

\textbf{Blayne Alaniz, Director of Student Disability Services and
Testing Services}\\
UMHB Box 8437\\
900 College Street\\
Belton, Texas 76513\\
Office: (254) 295-4739\\
Fax: (254) 295-4196\\
Email: \texttt{\href{mailto:balaniz@umhb.edu}{balaniz@umhb.edu}}

\subsection{Course Structure}

All assignments and other coursework are completed individually.
However, during and between certain class meetings you may either be
assigned to or asked to form small groups in order to collaborate on
data analysis projects and/or presentation(s) You will be guided through
the following course learning modules. See Section 1.5 for corresponding
student learning outcomes per learning module.

\subsubsection{Learning Modules}

EDAD 7312 is divided into four (4) learning modules:

\begin{enumerate}
\def\labelenumi{\arabic{enumi}.}
\tightlist
\item
  Introduction to Data
\item
  Foundations for Statistical Inference
\item
  Statistical Inference for Numerical and Categorical Data
\item
  Simple and Multiple Regression
\end{enumerate}

\subsection{Course Communication}

\subsubsection{Email}

Most all course communication outside of class will take place via
email. I will routinely email you course updates and announcements to
your UMHB-assigned email address. Thus, you should check your email
frequently. Likewise, due to the nature of this class and the
corresponding assignments, you will likely need to contact me with
questions. I am committed to responding as quickly as possible to your
questions via email. As a result, you can expect me to respond, on
average, within several hours of your email---often sooner. However, in
some circumstances, a personal visit during office hours or other
scheduled appointment may be more efficient than email. You are welcome
to call me on my office line: (254) 295-4553. This can be an even more
efficient method for quick troubleshooting inquiries.

\subsubsection{Remind}

There may be occasions when alerting you to course-related updates may
be most effective in real-time. In these situations, I will communicate
with you through a free, safe, and one-way messaging service called
Remind. To sign up for these alerts, text \texttt{@educ7312} to 81010
and follow the instructions. If you have trouble with this method, try
texting \texttt{@educ7312} to (254) 296-8301. Additionally, although not
likely, there may be extenuating circumstances which require me to delay
and/or cancel a class or other meeting.

\section{Course Requirements}

\subsection{Individual and Team Assignments}

\subsubsection{Journal Article Critique}

The purpose of this assignment is to read and examine critically all
sections of a peer-reviewed journal article. As a student of educational
research, it is imperative that you gain familiarity and comfort with
the structure/purpose of the scientific literature. \emph{A detailed
submission template and grading rubric will be provided}.

For each critique, you will select one quantitative research article
published within the last five years in a peer-reviewed journal. You are
free to select any article from any reputable, peer-reviewed journal
under the following conditions:

\begin{enumerate}
\def\labelenumi{\arabic{enumi}.}
\tightlist
\item
  The topic of research and theoretical framework are \emph{unrelated}
  to your
  dissertation.\footnote{The purpose of forcing you to branch outside of your own topic/theoretical framework is to allow you to, hopefully, experience different approaches to your own planned quantitative methodology. For example, various disciplines use a variety of nomenclature to describe the elements of research methods and design. Assume you have committed to learning how to play classical piano. All you listen to, practice, and perform are arrangements from the classical greats. However, imagine the perspective you might gain by listening to a jazz pianist improvise? The point is to expose you to various ways in which your methodology is implemented in other "genres" of research outside of you own. The more familiar you can be with your methodology now, the better off you will be both during your oral qualifying exam as well as your dissertation proposal presentation.}
\item
  However, the author(s) utilized quantitative methodology similar to or
  identical to that of your own planned research, as you currently
  understand it.
\end{enumerate}

Journal article critiques are due on the following dates:

\begin{enumerate}
\def\labelenumi{\arabic{enumi}.}
\tightlist
\item
  Sunday, February 09, 2020
\item
  Sunday, March 29, 2020
\end{enumerate}

\subsubsection{Special Topics Team Presentation}

This assignment consists of pairs of students presenting one or more
special topics from a section(s) of a single chapter from the textbook.
Teams should prepare a thorough lecture with accompanying slides,
handouts, etc. You should use no fewer than three additional academic
resources, not counting the textbook, to supplement your presentation.
Lectures should be organized, rigorous, and comprehensive. You should
assume the burden of responsibility for providing your peers everything
you can in order to ensure they have as complete an understanding about
your assigned topic(s) as you and your partner. They will do the same.
\emph{A detailed submission template and grading rubric will be
provided}.

Team numbers and student pairs were generated using a random sampling
permutation method. Based on the team number, chapter assignments and
dates were implemented by the instructor.

\begin{table}[H]
\begin{center}
\caption{Special Topics Team Presentation Dates}
\vspace{3mm}
\begin{tabular}{llll}
\hline
\textbf{Team} & \textbf{Students} & \textbf{Topics} & \textbf{Presentation Date} \\
\hline
\multirow{2}{*}{1} & Wendy Gamble & \multirow{2}{*}{Chapter 6} & \multirow{2}{*}{Friday, January 31, 2020} \\
 & Carmen Lewis &  &  \\
\hline
\multirow{3}{*}{2} & John Bate & \multirow{3}{*}{Chapter 7} & \multirow{3}{*}{Friday, January 31, 2020} \\
& Autumn Leal-Shopp &  &  \\
& Kemi Okafor &  &  \\
\hline
\multirow{3}{*}{3} & Lyndsae Benton & \multirow{3}{*}{Chapter 8} & \multirow{3}{*}{Friday, January 31, 2020} \\
& Mia Hall &  & \\
& Andrea Lail &  & \\
\hline
\multirow{3}{*}{4} & Dahlia Berwise & \multirow{3}{*}{Chapter 9} & \multirow{3}{*}{Friday, February 21, 2020} \\
& Mike McCarthy &  &  \\
& Dottie Jones &   & \\
\hline
\end{tabular}
\end{center}
\end{table}

\subsubsection{Individual Data Analysis Lab Assignments}

Individual lab assignments will feature a series of guided,
application-based, statistical analysis prompts that you will complete
using
IBM\textsuperscript{\textregistered}~SPSS\textsuperscript{\textregistered}~Statistics.
These assignments require you to apply what you learn within each of the
course modules. Each lab assignment will include a detailed set of
directions available on the course website. Additionally, you will be
provided a skeleton-like lab report template for each of the lab
assignments. More details on completing the lab assignments will be
provided in class. Table 2 below contains the topic and due date for
each lab assignment. \emph{Note}: These are individual assignments. All
elements of the academic integrity policy will be upheld.

\begin{table}[H]
\begin{center}
\caption{Individual Lab Assignment Dates}
\label{labs}
\vspace{3mm}
\begin{tabular}{llr}
\hline
 & {\bf{Topic}} & {\bf{Due}}\\
\hline
1. & Exploring and Assessing Relationships Between Variables & 02-16-2020\\
2. & Basic and Advanced Concepts in Predictive Modeling & 03-08-2020\\
3. & Making Advanced Statistical Inferences Between and Within Groups & 04-05-2020\\
\hline
\end{tabular}
\end{center}
\end{table}

\subsubsection{Chapter 3 in a Nutshell}

Your culminating assignment/project in EDAD 7324 will be comprised of an
abridged, rough sketch version of your dissertation's methods section
(chapter 3). In chapter 3, you detail, in a sense, the who, what, when,
where, and hows of your particular experiment, study, or project. In the
case of a quantitative dissertation, this involves, at minimum:

\begin{enumerate}
\def\labelenumi{\arabic{enumi}.}
\tightlist
\item
  details about your participants,
\item
  sampling procedures,
\item
  random assignment methods, (if applicable),
\item
  any materials used (e.g., instruments, software, questionnaires,
  etc.),
\item
  definitions and identification of all variables (e.g., independent
  vs.~dependent; predictor vs.~outcome; explanatory vs.~response, etc.),
  and finally,
\item
  the statistical methods used in order to test hypotheses or make
  inferences.
\end{enumerate}

I realize this assignment may feel a little daunting for some of you.
Not to worry. Again, the more familiar you can be with your methodology
now, the better off you will be both during your oral qualifying exam as
well as your dissertation proposal presentation. This assignment will
help prepare you for Saturday, April 18! \emph{A detailed submission
template and grading rubric will be provided}.

Chapter 3 in a Nutshell is due \textbf{Sunday, April 05, 2020.}

\subsection{Grade Calculation}

\subsubsection{Individual and Team Performance}

Table 3 below lists all assignments, their point value, and proportion
of weighted total. See Table 4 for final grade calculation and letter
grade distribution.

\begin{table}[H]
\centering
\caption{Individual Assignments and Point Values}
\vspace{3mm}
\label{points}
\begin{tabular}{lllrlrr}
\hline
\bf{Assignment} & \bf{\em{n}} &  & \bf{Points} &  & \bf{Total} & \bf{Prop.} \\
\hline
Journal Article Critiques & 2 & $\times$ & 100 & = & 200 & .30\\
Special Topics Team Presentations & 1 & $\times$ & 100 & = & 100 & .20\\
Individual Lab Assignments & 3 & $\times$ & 10 & = & 30 & .20 \\
Chapter 3 in a Nutshell & 1 & $\times$ & 100 & = & 100 & .25 \\
Attendance & 5 & $\times$ & 1 & = & 5 & .05 \\
\multicolumn{4}{r}{\bf{Individual Performance Total}} & {\bf{=}} & {\bf{435}} & {\bf{1.00}} \\
\hline
\end{tabular}
\label{points}
\end{table}

\subsubsection{Final Grade Calculation}

All course grades will be posted in the gradebook in myCourses. All
point totals and proportional weights listed in Table 2 are reflected in
myCourses. Thus, your current grade in myCourses should reflect your
actual grade. Table 4 below describes the point range required to
achieve a given letter grade.

\begin{table}[H]
\begin{center}
\caption{Final Grade Point Range Requirements}
\label{finalgrades}
\vspace{3mm}
\begin{tabular}{cccc}
\hline
\bf{Grade} & \bf{Point Range} & \bf{Percentage} & \bf{Grade Points}\\
\hline
A & 391.50 -- 435.00 & 90 -- 100 & 4.0\\ 
B & 348.00 -- 232.00 & 80 -- 89  & 3.0\\ 
C & 304.50 -- 347.00 & 70 -- 79  & 2.0\\ 
D & 261.00 -- 303.50 & 60 -- 69  & 1.0\\ 
F & 000.00 -- 260.00 & 00 -- 59  & 0.0\\
\hline
\end{tabular}
\end{center}
\end{table}

\section{Policies}

\subsection{Attendance}

Your regular attendance and participation in this course is expected. I
will record and maintain attendance records for each student. Attendance
is worth 5\% of your final grade. In other words, if you attend 050\% of
the scheduled class meetings you will earn the complete 5\% attendance
total. Any University- or otherwise-excused absence will not count
toward this total. At the conclusion of the semester, the percentage of
class meetings you attended will be multiplied by 0.05 to obtain your
attendance grade.

\subsection{Late Work}

All assignments are considered late if submitted after the date and time
specified in the syllabus and/or course website. This policy will be
enforced in the event that assignment deadlines are revised during the
course of the term. Assignments submitted late will result in a penalty
of 20 percentage points per day.\textbackslash{}

For example, if an assignment is due on March 22, 2020 and is submitted
within 24 hours of the due date and time that assignment will result in
an automatic deduction of 20 percentage points from the assignment raw
score. In other words, if you submit an assignment worth 10 points on
March 23, 2020, and the assignment was originally due March 22, 2020,
and you score a 9.5/10, then your new score would be:

\begin{equation}
9.5 - (9.5)(0.20) \times 100 = 7.6.
\end{equation}

Assignments submitted more than five calendar days late will receive a
grade of zero. To ensure fairness, this policy will be strictly
enforced. Exceptons are made at the discretion of the instructor and may
include, but are not limited to:

\begin{enumerate}
\def\labelenumi{\arabic{enumi}.}
\tightlist
\item
  Death in the immediate family (parent, spouse, sibling, child)
\item
  Unforeseeable medical emergency affecting yourself, your spouse, or
  your child (e.g., automobile accident, major sickness, et al.).
\item
  Participation in an official UMHB-sponsored event
\end{enumerate}

\emph{Note}: Routine medical appointments or clinical visits related to
minor illnesses do not qualify as an unforeseeable medical emergency.
Likewise, conflicts with a work schedule or trips not related to
official UMHB events do not qualify for assignment absolution.

\section{Disclaimer}

Syllabus is subject to change at instructor's discretion.

\newpage
\begin{landscape}

\section{Course Calendar}

\begin{tabularx}{\linewidth}{XlXXr}
\hline
\textbf{Module} & \textbf{Week} & \textbf{Date} & \textbf{Topic} & \textbf{Reading} \\
\hline
A Gentle Introduction to Advanced Statistics in Research & 1 & Friday, January 10, 2020 & Central Tendency, Variability, and Descriptive Statistics & Ch. 03 \\
 &  &  & Z-Scores and the Area Under the Normal Curve & Ch. 04 \\
 &  &  & Statistical Significance, Confidence Intervals, and Effect Size & Ch. 05 \\
 &  &  &  &  \\
Exploring and Assessing Relationships Between Variables & 2 & Friday, January 31, 2020 & Bivariate Correlation & Ch. 06 \\
 &  &  & Bivariate Regression & Ch. 07 \\
 &  &  & Partial Correlation and Statistical Control & Ch. 08 \\
 &  &  &  &  \\
Basic and Advanced Concepts in Predictive Modeling & 3 & Friday, February 21, 2020 & Multiple Regression I: Basic Concepts & Ch. 09 \\
 &  &  & Multiple Regression II: Advanced Concepts & Ch. 10 \\
 &  &  &  &  \\
Making Advanced Statistical Inferences Between and Within Groups & 4 & Friday, March 20, 2020 & Analyses of Variance (ANOVA) and Multivariate Analyses of Variance (MANOVA) & Ch. 13 \\
 &  &  & Analyses of Covariance (ANCOVA) & Ch. 14 \\
 &  &  &  &  \\
--- & 5 & Saturday, April 18, 2020 & Oral Qualifying Exam & --- \\
\hline
\end{tabularx}

\end{landscape}

\newpage
\section{Calendar of Due Dates}

\begin{tabularx}{\linewidth}{p{4cm}Xr}
\hline
\textbf{Month} & \textbf{Date} & \textbf{Due} \\
\hline
\multirow{3}{*}{January} & Friday, January 31, 2020 & Team 1 Presentation \\
& Friday, January 31, 2020 & Team 2 Presentation \\
& Friday, January 31, 2020 & Team 3 Presentation \\
\hline
&  &  \\

\multirow{3}{*}{February} & Sunday, February 09, 2020 & Article Critique \#1 \\
& Sunday, February 16, 2020 & Lab 01 \\
& Friday, February 21, 2020 & Team 4 Presentation \\
\hline
&  &  \\

\multirow{2}{*}{March} & Sunday, March 08, 2020 & Lab 02 \\
& Sunday March 29, 2020 & Article Critique \#2 \\
\hline
&  &  \\

\multirow{2}{*}{April} & Sunday, April 05, 2020 & Lab 03 \\
& Sunday, April 05, 2020 & Chapter 3 in a Nutshell \\
\hline
\end{tabularx}

\end{document}
